\documentclass{article}
\textheight 590pt
\textwidth 6.7in
\evensidemargin 0pt
\oddsidemargin 0pt
\topmargin 0pt
\pagestyle{empty}
\usepackage{graphicx}
\usepackage{amsmath}

\begin{document}
\begin{center}
AA 543 Homework 1  \\
Due Fri, Jan 13, 2012 \\
Jonathan Varkovitzky
\end{center}

\newcommand{\ul}{\underline}
\newcommand{\p}{\partial}


\begin{enumerate}
%*************************
\item \textbf{Question 1:}
\begin{enumerate}
%******************
\item \textbf{1.1:}
The three equation for conservation of mass, momentum and energy are given below:

\begin{enumerate}
\item Conservation of Mass:
\begin{equation}
\boxed{\frac{\partial \rho}{\partial t} + \underline{\nabla}\cdot(\rho \underline{u}) = 0}
\end{equation}

\item Conservation of Momentum:
\begin{equation}
\boxed{\frac{\partial(\rho\underline{u})}{\partial t} + \underline{\nabla}\cdot(\rho \underline{u}\underline{u}) = \underline{\nabla}\underline{\sigma}+\rho \underline{F} }
\end{equation}

\item Conservation of Energy:
\begin{equation}
\boxed{\frac{\partial(\rho E)}{\partial t} + \underline{\nabla}\cdot(\rho\underline{u}E)=\rho \dot{q}+\underline{\nabla}\cdot(k\underline{\nabla}T)+\rho\underline{F}\cdot\underline{u}-\underline{\nabla} \cdot (p \underline{u})+\underline{\nabla}\cdot(\underline{\tau}\cdot\underline{u})}
\end{equation}
\end{enumerate}
%******************
\item \textbf{1.2:}
We now derive the non-conservative form of the three equations above:

\begin{enumerate}

\item Mass:
\begin{equation}
\frac{\p \rho}{\p t} + \ul{\nabla}\cdot(\rho\ul{u}) = 0
\label{mass1}
\end{equation}
Now we expand the $\ul{\nabla}\cdot(\rho\ul{u})$ term:
\begin{equation}
\ul{\nabla}\cdot(\rho\ul{u}) = \rho \ul{\nabla}\cdot \ul{u} + \ul{u}\cdot\ul{\nabla} \rho
\end{equation}
Plugging this into (\ref{mass1}) we get:
\begin{equation}
\frac{\p \rho}{\p t} + \rho \ul{\nabla}\cdot \ul{u} + \ul{u}\cdot\ul{\nabla} \rho = 0
\label{mass2}
\end{equation}
We now use the definition of material derivatives $(\frac{D\rho}{Dt} = \frac{\p \rho}{\p t} + \ul{u}\cdot\ul{\nabla}\rho)$ and plug into (\ref{mass2}) to get the non-conservative form of mass conservation:
\begin{equation}
\boxed{\frac{D \rho}{Dt} + \rho \ul{\nabla}\cdot\ul{u} = 0}
\end{equation}


\item Momentum:
\begin{equation}
\frac{\p (\rho \ul{u})}{\p t} + \ul{\nabla}\cdot(\rho \ul{u}\ul{u}) = \ul{\nabla}\ \ul{\ul{\sigma}}+\rho\ul{F}
\label{momentum}
\end{equation} 

Expanding the $\ul{\nabla}\cdot(\rho \ul{u}\ul{u})$ term gives us:

\begin{equation}
\ul{\nabla}\cdot(\rho \ul{u}\ul{u}) = \rho \ul{u}\cdot(\nabla\ul{u}) + \ul{u}\ul{\nabla}\cdot(\rho\ul{u})
\label{expand1}
\end{equation}

Additionally, by multiplying the continuity equation by $\ul{u}$ we get:

\begin{equation}
\ul{u} \frac{\p \rho}{\p t} + \ul{u}\ul{\nabla}\cdot(\rho \ul{u})
\label{ucontinuity}
\end{equation}


We expand $\frac{\p (\rho\ul{u})}{\p t}$ using product rule to get:
\begin{equation}
\frac{\p (\rho\ul{u})}{\p t} = \ul{u}\ \frac{\p \rho}{\p t} + \rho \frac{\p \ul{u}}{\p t}
\label{prodRule}
\end{equation}

Plugging in equations (\ref{expand1}) and (\ref{prodRule}) into equation (\ref{momentum}) we get:

\begin{equation}
\ul{u} \frac{\p \rho}{\p t} + \rho \frac{\p \ul{u}}{\p t} + \rho \ul{u} \cdot (\nabla \ul{u}) + \ul{u}\ul{\nabla}\cdot(\rho \ul{u}) = \ul{\nabla}\cdot \ul{\ul{\sigma}} + \rho \ul{F}
\label{bigMess}
\end{equation}

We now subtract equation (\ref{ucontinuity}) from equation(\ref{bigMess}) to give us:

\begin{equation}
\rho \frac{\p \ul{u}}{\p t} + \rho \ul{u} \cdot(\nabla \ul{u}) = \ul{\nabla}\cdot\ul{\ul{\sigma}} + \rho \ul{F}
\label{almostThere}
\end{equation}

We now finally need the definition of a material derivative:

\begin{equation}
\frac{D\ul{u}}{Dt} = \frac{\p \ul{u}}{\p t} + \ul{u}\cdot\ul{\nabla}\ul{u}
\end{equation}

Applying this to equation (\ref{almostThere}) we get the non-conservative form of momentum conservation:

\begin{equation}
\boxed{\rho \frac{D\ul{u}}{Dt} = \ul{\nabla}\cdot\ul{\ul{\sigma}} + \rho \ul{F}}
\end{equation}

\item Energy:

\begin{equation}
\frac{\partial(\rho E)}{\partial t} + \underline{\nabla}\cdot(\rho\underline{u}E)=\rho \dot{q}+\underline{\nabla}\cdot(k\underline{\nabla}T)+\rho\underline{F}\cdot\underline{u}-\underline{\nabla} \cdot (p \underline{u})+\underline{\nabla}\cdot(\underline{\tau}\cdot\underline{u})
\label{energy}
\end{equation}

We can combine the last two terms of equation (\ref{energy}) using the following relation:

\begin{equation}
-\ul{\nabla}\cdot(\rho\ul{u}) + \ul{\nabla}\cdot(\ul{\ul{\tau}}\cdot\ul{u}) = \ul{\nabla}\cdot(\ul{\ul{\sigma}}\cdot\ul{u})
\label{sigmaEqn}
\end{equation}

We can now simplify equation (\ref{energy}) by the relation in equation (\ref{sigmaEqn}):

\begin{equation}
\frac{\partial(\rho E)}{\partial t} + \underline{\nabla}\cdot(\rho\underline{u}E)=\rho \dot{q}+\underline{\nabla}\cdot(k\underline{\nabla}T)+\rho\underline{F}\cdot\underline{u}+\ul{\nabla}\cdot(\ul{\ul{\sigma}}\cdot\ul{u})
\label{energySimple}
\end{equation}

We now integrate equation (\ref{energySimple}) over an arbitrary volume which will allow us to use certain relations later:

\begin{equation}
\int_V \frac{\partial(\rho E)}{\partial t} + \underline{\nabla}\cdot(\rho\underline{u}E)\ dV = \int_V \rho \dot{q}\ dV + \int_V \underline{\nabla}\cdot(k\underline{\nabla}T)\ dV + \int_V \rho\underline{F}\cdot\underline{u}\ dV + \int_V\ul{\nabla}\cdot(\ul{\ul{\sigma}}\cdot\ul{u})\ dV
\label{intEnergySimple}
\end{equation}

We now convert the left hand side into a material derivative via the Reynolds Transport Theorem:

\begin{equation}
\int_V \frac{\partial(\rho E)}{\partial t} + \underline{\nabla}\cdot(\rho\underline{u}E)\ dV = \frac{D}{Dt}\int_V \rho E\ dV
\label{RTT}
\end{equation}

Additionally, we use Gauss' Theorem to covert the following terms:

\begin{eqnarray}
&&\int_V \underline{\nabla}\cdot(k\underline{\nabla}T)\ dV = \oint_S k\cdot \ul{n}\ dS\nonumber \\
&&\int_V\ul{\nabla}\cdot(\ul{\ul{\sigma}}\cdot\ul{u})\ dV = \oint_S (\ul{\ul{\sigma}}\cdot\ul{u})\cdot\ul{n}\ dS
\label{gauss}
\end{eqnarray}

Substituting equations (\ref{RTT} \& \ref{gauss}) into equation (\ref{intEnergySimple}) we get the non-conservative form of energy conservation:

\begin{equation}
\boxed{\frac{D}{Dt}\int_V \rho E\ dV = \int_V \rho \dot{q}\ dV + \oint_S k\cdot \ul{n}\ dS + \int_V \rho\underline{F}\cdot\underline{u}\ dV + \oint_S (\ul{\ul{\sigma}}\cdot\ul{u})\cdot\ul{n}\ dS}
\end{equation}


\end{enumerate}

\end{enumerate}


%*************************
\item \textbf{Question 2:}
\begin{enumerate}
%******************
\item \textbf{2.1:}
We begin with the 3-D momentum equation for a viscous, unsteady, and compressible flow.
\begin{equation}
\rho \frac{Du}{Dt} = \underline{\nabla}\cdot\underline{\sigma}+\rho \underline{F}
\end{equation}
We now can derive the 1-D equation by expanding the $\sigma$ term and converting the $\nabla$'s into derivatives strictly in x. To give us:

\begin{eqnarray}
&& \rho \frac{\partial u}{\partial t} + \rho u \frac{\partial u}{\partial t} = \frac{\partial}{\partial x}\left[\mu(2\frac{\partial u}{\partial x})-\frac{2}{3}\mu\frac{\partial u}{\partial x} \right] + \rho F - \frac{\partial p}{\partial x}\nonumber \\
&& \boxed{\rho \frac{\partial u}{\partial t} + \rho u \frac{\partial u}{\partial t} = \frac{4}{3}\mu\frac{\partial^2 u}{\partial x^2} + \rho F - \frac{\partial p}{\partial x}}
\end{eqnarray}

%******************
\item \textbf{2.2:}
Dropping the pressure gradient term and dividing the equation by $\rho$ we get:
\begin{equation}
\frac{\partial u}{\partial t} + u \frac{\partial u}{\partial t} = \frac{4}{3}\nu\frac{\partial^2 u}{\partial x^2} +  F,\ \nu = \frac{\mu}{\rho}
\end{equation}
We can now reduce this second order equation into a first order system:

\begin{equation}
  \left\{\begin{array}{l}
    \frac{\partial u}{\partial x} = v\\
    \frac{\partial u}{\partial t} + u \frac{\partial u}{\partial x} - \frac{4}{3}\nu\frac{\partial v}{\partial x} = F \end{array}\right.
\end{equation}

We now convert this into matrix form and look at the homogeneous component:

\begin{equation}
\left[\begin{array}{cc} 1 & 0 \\ 0 & 0\end{array} \right] \frac{\partial}{\partial t}\left[\begin{array}{c}u \\ v \end{array} \right] + \left[\begin{array}{cc}u & \frac{-4}{3}\nu \\ 1 & 0  \end{array} \right]\frac{\partial}{\partial x}\left[\begin{array}{c}u \\ v \end{array} \right]=0
\end{equation}

Allowing $m_1 = \frac{\partial}{\partial t}\left[\begin{array}{c}u \\ v \end{array} \right]$ and $m_2 = \frac{\partial}{\partial x}\left[\begin{array}{c}u \\ v \end{array} \right]$ we take the determinant of the system and set it to zero

\begin{eqnarray}
&& \left|\left[\begin{array}{cc} 1 & 0 \\ 0 & 0\end{array} \right] m_1 + \left[\begin{array}{cc}u & \frac{-4}{3}\nu \\ 1 & 0  \end{array} \right] m_2 \right|=0 \nonumber \\
&& \left|\left[\begin{array}{cc} m_1 + m_2 u & \frac{-4}{3}\nu m_2  \\ m_2 & 0\end{array} \right]\right| = 0\nonumber \\
&& \Rightarrow m_2 = 0 
\end{eqnarray}

This implies that the matrix is rank 1, since this is not full rank it implies that the system is \fbox{parabolic}.

\end{enumerate}


%*************************
\item \textbf{Question 3:}
\begin{enumerate}
%******************
\item \textbf{3.1:}
We derive the 1-D inviscid flow equations from the full 3-D NS equations:

\begin{enumerate}
\item Mass: No reduction is needed:

\begin{equation}
\boxed{\frac{\p \rho}{\p t} + \frac{\p (\rho u)}{\p x} = 0}
\end{equation}

\item Momentum:

\begin{eqnarray}
&&\frac{\p (\rho u)}{\p t} + \frac{\p (\rho u^2)}{\p x} = -\frac{\p p}{\p x} + \frac{\p \tau}{\p x} + \rho F\nonumber \\
&& \Rightarrow \boxed{\frac{\p (\rho u)}{\p t} + \frac{\p (\rho u^2)}{\p x} = -\frac{\p p}{\p x} + \rho F}
\end{eqnarray}

\item Energy:
We used equation (1.4.14) from the text:
\begin{equation}
\boxed{\frac{\p (\rho e)}{\p t} + \frac{\p (\rho u h)}{\p x} = u \frac{\p p}{\p x} + \frac{\p}{\p x}\left(k \frac{\p T}{\p x} \right)+\epsilon_V + q_H}
\label{energyEqn}
\end{equation}

\end{enumerate}
%******************
\item \textbf{3.2:}

Since we are assuming the fluid is isentropic we know that $\frac{\p s}{\p T} = 0$ and hence use the following relation to cancel out the last three terms in the equation:

\begin{equation}
\rho T \frac{\p s}{\p T} = \epsilon_V+\nabla\cdot(k\nabla T)+q_H
\end{equation}

Plugging this into equation (\ref{energyEqn}):
\begin{equation}
\frac{\p (\rho e)}{\p t} + \frac{\p (\rho u h)}{\p x} = u \frac{\p p}{\p x}
\label{justABitMore}
\end{equation}

Now we use the following definitions to reduce equation (\ref{justABitMore}):

\begin{eqnarray}
&&h = c_pT = \frac{\gamma}{\gamma-1} \frac{p}{\rho}\nonumber \\
&&e = \frac{1}{\gamma-1} \frac{p}{\rho}\nonumber \\
\end{eqnarray}

This gives us:

\begin{eqnarray}
&& \frac{1}{\gamma-1}\frac{\p p}{\p t} + \frac{\gamma}{\gamma-1} \frac{\p (up)}{\p x} = u \frac{\p p}{\p x}\nonumber \\
&& \frac{1}{\gamma-1}\frac{\p p}{\p t} + \left(\frac{\gamma}{\gamma-1}-1\right)u \frac{\p p}{\p x} + \frac{\gamma}{\gamma-1}p\frac{\p u}{\p x}= 0\nonumber \\
&& \frac{\p p}{\p t} + u \frac{\p p}{\p x} = -\gamma p \frac{\p u}{\p x} \nonumber \\
&& \Rightarrow \boxed{\frac{\p p}{\p t} + u \frac{\p p}{\p x} = -\rho c^2 \frac{\p u}{\p x}}
\end{eqnarray}

%******************
\item \textbf{3.3:}
Before we can determine the eigenvalues and type of the system of PDEs we must first expand a number of terms from the equations laid out previously in this question using product rule and additionally drop any non-homogeneous terms.  

\begin{enumerate}
\item Mass:
\begin{eqnarray}
&& \frac{\p \rho}{\p t} + \frac{\p (\rho u)}{\p x} = 0 \nonumber \\
&& \Rightarrow \frac{\p \rho}{\p t} + u \frac{\p \rho}{\p x} + \rho \frac{\p u}{\p x}= 0
\end{eqnarray}

\item Momentum:
\begin{eqnarray}
&& \frac{\p (\rho u)}{\p t} + \frac{\p (\rho u^2)}{\p x} = -\frac{\p p}{\p x} \nonumber \\
&& \Rightarrow u \frac{\p \rho }{\p t} + \rho \frac{\p u }{\p t} + u \frac{\p (\rho u) }{\p x} + \rho u \frac{\p u }{\p x} = -\frac{\p p }{\p x}\nonumber \\
&& \Rightarrow \rho \frac{\p u }{\p t} + \rho u \frac{\p u }{\p x} + \frac{\p p }{\p x} = 0
\end{eqnarray}

\item Energy:
As solved for in part 3.2:
\begin{equation}
\frac{\p p}{\p t} + u \frac{\p p}{\p x} + \rho c^2 \frac{\p u}{\p x} = 0
\end{equation}
\end{enumerate}


We now take the above equations and need to put them into a vector form, this yields the following system:

\begin{equation}
\frac{\p}{\p t} \left[\begin{array}{c} \rho \\ u \\ p \end{array}\right]+\left[\begin{array}{ccc} u & 1 & 0 \\ 0 & u & \frac{1}{\rho}\\0 & \rho c^2 & u \end{array} \right]\frac{\p}{\p x} \left[\begin{array}{c} \rho \\ u \\ p \end{array}\right]
\end{equation}

This system is of the form $\frac{\p \vec{u}}{\p t} + A \frac{\p \vec{u}}{\p x} = 0$, which impels that we want a solution of the form $\vec{u} = \hat{u} e^{I(kx-\omega t)}$. Hence the A matrix contains the eigenvalue information we need to use to determine the classification of the system.

\begin{eqnarray}
&& 0 = det \left|\begin{array}{ccc} u & 1 & 0 \\ 0 & u & \frac{1}{\rho}\\0 & \rho c^2 & u \end{array} \right| = (uk - \omega)\left[(uk-\omega)^2-k^2c^2 \right]\nonumber \\
&& \Rightarrow \boxed{\omega_1 = uk}\nonumber \\
&& \Rightarrow (uk-\omega)^2-k^2c^2 = 0 \nonumber \\
&& \Rightarrow \omega^2 - \omega 2uk+(uk)^2-(kc)^2 = 0\nonumber \\
&& \Rightarrow \omega_{2,3} = \frac{2uk}{2} \pm \frac{\sqrt{4(uk)^2-4(uk)^2+k^2c^2}}{2}\nonumber \\
&& \Rightarrow \boxed{ \omega_{2,3} = uk \pm kc}
\end{eqnarray}

The fact that we have three eigenvalues that are real implies that the system is \fbox{hyperbolic}.

\end{enumerate}

\end{enumerate}
\end{document}
